\documentclass[journal]{IEEEtran}
\usepackage{cite}
\usepackage{amsmath}
\usepackage{algorithmic}
\usepackage{array}

% *** SUBFIGURE PACKAGES ***
\ifCLASSOPTIONcompsoc
 \usepackage[caption=false,font=normalsize,labelfont=sf,textfont=sf]{subfig}
\else
 \usepackage[caption=false,font=footnotesize]{subfig}
\fi

% correct bad hyphenation here
\hyphenation{op-tical net-works semi-conduc-tor}


\begin{document}
\title{A survey on vision transformer}

\author{Yuhuan Yang,~\IEEEmembership{MedianBrainGroup,STJU,}}

\maketitle

\begin{abstract}
The abstract goes here.
\end{abstract}

\begin{IEEEkeywords}
IEEE, IEEEtran, journal, \LaTeX, paper, template.
\end{IEEEkeywords}

\IEEEpeerreviewmaketitle

\section{Introduction\cite{w}}
Active contour model, also called snakes, is a framework in computer vision introduced by Michael Kass, Andrew Witkin, and Demetri Terzopoulos\cite{snake} for delineating an object outline from a possibly noisy 2D image. The snakes model is popular in computer vision, and snakes are widely used in applications like object tracking, shape recognition, segmentation, edge detection and stereo matching.

A snake is an energy minimizing, deformable spline influenced by constraint and image forces that pull it towards object contours and internal forces that resist deformation. Snakes may be understood as a special case of the general technique of matching a deformable model to an image by means of energy minimization.\cite{snake} In two dimensions, the active shape model represents a discrete version of this approach, taking advantage of the point distribution model to restrict the shape range to an explicit domain learnt from a training set.

Snakes do not solve the entire problem of finding contours in images, since the method requires knowledge of the desired contour shape beforehand. Rather, they depend on other mechanisms such as interaction with a user, interaction with some higher level image understanding process, or information from image data adjacent in time or space.


\section{Introduction}

\IEEEPARstart{T}{his} demo file is intended to serve as a ``starter file''
for IEEE journal papers produced under \LaTeX\ using
IEEEtran.cls version 1.8b and later.

\subsection{Subsection Heading Here}
Subsection text here.

\subsubsection{Subsubsection Heading Here}
Subsubsection text here.


%\begin{figure}[!t]
%\centering
%\includegraphics[width=2.5in]{myfigure}
%\caption{Simulation results for the network.}
%\label{fig_sim}
%\end{figure}

%\begin{figure*}[!t]
%\centering
%\subfloat[Case I]{\includegraphics[width=2.5in]{box}%
%\label{fig_first_case}}
%\hfil
%\subfloat[Case II]{\includegraphics[width=2.5in]{box}%
%\label{fig_second_case}}
%\caption{Simulation results for the network.}
%\label{fig_sim}
%\end{figure*}
%

\begin{table}[!t]
%% increase table row spacing, adjust to taste
\renewcommand{\arraystretch}{1.3}
% if using array.sty, it might be a good idea to tweak the value of
% \extrarowheight as needed to properly center the text within the cells
\caption{An Example of a Table}
\label{table_example}
\centering
%% Some packages, such as MDW tools, offer better commands for making tables
%% than the plain LaTeX2e tabular which is used here.
\begin{tabular}{|c||c|}
\hline
One & Two\\
\hline
Three & Four\\
\hline
\end{tabular}
\end{table}

\section{Conclusion}
The conclusion goes here.\cite{han2020survey}

% Can use something like this to put references on a page
% by themselves when using endfloat and the captionsoff option.
\ifCLASSOPTIONcaptionsoff
  \newpage
\fi

% references section
\bibliographystyle{IEEEtran}
\bibliography{IEEEabrv,../bib/paper}

\end{document}


